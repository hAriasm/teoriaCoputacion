\chapter{Resultados y/o Evaluaciones}
\hrule \bigskip \vspace*{1cm}
%------------------------------------------------------------------------

\section{Conclusiones}
\begin{itemize}
  \item El lenguaje natural (LN) nos permite el designar las cosas actuales y razonar acerca de ellas, fue desarrollado y organizado a partir de la experiencia humana y puede ser utilizado para analizar situaciones altamente complejas y razonar muy sutilmente. 
 \item  Los lenguajes de programación (LP) son un tipo muy limitado de lenguaje natural, orientados básicamente a la manipulación de datos e información discreta, pero no son suficientes para la comunicación integral que incluya la totalidad de los aspectos semánticos y pragmáticos.
 \item  El procesamiento de lenguaje natural (PLN) consiste en la utilización de un lenguaje natural para comunicarnos con la computadora, debiendo esta entender las oraciones que le sean proporcionadas. El uso de estos lenguajes naturales facilita el desarrollo de programas que realicen tareas relacionadas con el lenguaje o bien, desarrollar modelos que ayuden a comprender los mecanismos humanos relacionados con el lenguaje. Los lexicones son una parte importante del procesamiento de lenguaje natural y debe contener información fonológica, morfológica, sintáctica, semántica y pragmática, pero además esta información debe ser estructurada de forma que permita su reutilización para diversas tareas.
 \item  El lexicón también tiene que incluir otros tipos de información que considere aspectos de orden idiosincrática, de pronunciación, y toda información que no se puede derivar del significado de las palabras o de su forma morfológica.
\end{itemize}

\section{Contribuciones}

Con todo lo que has investigado, propuesto y/o desarrollado que haz
conseguido obtener para cooperar con la solución del problema.

\section{Trabajo futuro}

A partir del conocimiento generado en disciplinas como la informatica y la lingüística computacional, se están desarrollando sistemas para la confección de resúmenes y la indización automática. Este tipo de investigaciones se lleva practicando desde hace tiempo, y se comienza a recoger los frutos de años de inspección, por lo que se debe permanecer atentos a su evolución. El procesamiento del lenguaje natural es una labor
compleja, no exento de dificultad para los lingüísticas que deben adquirir la instrumentación de los informáticos, y para los informáticos, ya que deben hacer suyos 
conocimientos lingüísticos.
