\chapter{Introducción}
\hrule \bigskip \vspace*{1cm}
%------------------------------------------------------------------------
\section{Contexto y Motivación}
El desarrollo de aplicaciones con procesamiento de lenguaje natural abre líneas a varias posibles aplicaciones como chat boot, interpretación de textos automáticamente, desarrollo de aplicaciones de alto nivel entre otras. Para esto vimos la necesidad para que las aplicaciones la interpretación de un contexto puede cambiar completamente el sentido del contexto a una oración, para esto debemos reducir la ambigüedad en textos. 

\section{Definición del problema}
La ambigüedad se refiere a términos que son estructuras gramaticales que pueden entenderse de diferentes maneras o abrirse a diferentes interpretaciones y, por lo tanto, crean dudas, incertidumbre o confusión según \cite{EvaluacionAmbiguedad01}.



\section{Justificación}

Existe muchas justificaciones para justificar este tipo de investigación por ejemplo en el articulo de \cite{ChatBoot} no explica de la importancia de un chatboot en época de pandemia


\section{Objetivos}

 Analizar los métodos de reducción de ambigüedad semántica.

\section{Objetivos específicos}

\section{Organización de la tesis}

Una breve descripción de cada uno de los capítulos que estas
desarrollando desde el CAP 2 hasta el capitulo antes del apéndice.
