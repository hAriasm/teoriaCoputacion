\chapter{Introducción}
\hrule \bigskip \vspace*{1cm}
%------------------------------------------------------------------------
\section{Contexto y Motivación}
El desarrollo de aplicaciones con procesamiento de lenguaje natural abre varias posibles aplicaciones como chat boot, interpretación de textos automáticamente, desarrollo de aplicaciones de alto nivel, clasificación de textos, comparación de textos, intencionalidad, encriptación de documentos y otros mas proyectos. Es por eso que concluimos la necesidad para que las aplicaciones  de un contexto puede cambiar completamente el sentido del  a una oración, para esto debemos reducir la ambigüedad en textos.

\section{Definición del problema}
La ambigüedad se refiere a términos que son estructuras gramaticales que pueden entenderse de diferentes maneras o abrirse a diferentes interpretaciones y, por lo tanto, crean dudas, incertidumbre o confusión según \cite{EvaluacionAmbiguedad01}.

\section{Justificación}
La necesidad según \cite{Arquitectura} la utilización de PLN incremento en varios rubros y
existe muchas justificaciones para justificar este tipo de investigación por ejemplo en el articulo de \cite{ChatBoot} no explica de la importancia de un chatboot en época de pandemia


\section{Objetivos}

 Analizar la importancia de desambiguación en algoritmo de Procesamiento de Lenguaje Natural (PLN).

\section{Objetivos específicos}
Para poder alcanzar este objetivo es necesario los siguientes trabajos de investigación:
\begin{itemize}
  \item Demostrar que los algoritmos tiene una alta incertidumbre y necesitan mejoras para reducir la ambigüedad.
  \item Identificar y analizar el estado actual de los métodos usados para reducir la desambiguación.
\end{itemize}

\section{Organización de la tesis}

En en el capitulo 2 estaremos desarrollando el marco teórico, el capitulo 3 damos a conocer nuestra motivacion y el marco de trabajo que usamos,capitulo 4 mostramos un experimento para justificar nuestra investigación y en el capitulo 5 mostramos nuestras conclusiones y trabajos futuros
