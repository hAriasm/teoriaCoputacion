\chapter{Introducción}
\hrule \bigskip \vspace*{1cm}
%------------------------------------------------------------------------
\section{Contexto y Motivación}
El desarrollo de aplicaciones con procesamiento de lenguaje natural (PLN) hace posible la implementación de aplicaciones como chat boot, interpretación de textos automáticamente, desarrollo de aplicaciones de alto nivel, clasificación de textos, comparación de textos, intencionalidad, encriptación de documentos, entre otros. Por tal motivo consideramos importante analizar el problema de ambiguación en textos e identificar los métodos de desambiguación. 

\section{Definición del problema}
La ambigüedad se refiere a términos que son estructuras gramaticales que pueden entenderse de diferentes maneras o abrirse a diferentes interpretaciones y, por lo tanto, crean dudas, incertidumbre o confusión según \cite{EvaluacionAmbiguedad01}.

\section{Justificación}
En las investigaciones \cite{ChatBoot}, \cite{055} , \cite{056}se muestran las dificultades al existir ambiguación en textos en el Procesamiento de Lenguaje Natural, generando problemas de rendimiento y errores en las aplicaciones desarrolladas. Por las razones mencionadas, amerita el análisis de ambiguación en textos, con la finalidad de ser base a futuros proyectos de investigación y/o minimización a este problema. 

\section{Objetivos}

 Analizar la importancia de desambiguación en algoritmo de Procesamiento de Lenguaje Natural (PLN).

\section{Objetivos específicos}
Para poder alcanzar este objetivo es necesario los siguientes trabajos de investigación:
\begin{itemize}
  \item Demostrar que los algoritmos tiene una alta incertidumbre y necesitan mejoras para reducir la ambigüedad.
  \item Identificar y analizar el estado actual de los métodos usados para reducir la desambiguación.
\end{itemize}

\section{Organización de la tesis}

En en el capitulo 2 estaremos desarrollando el marco teórico, el capitulo 3 damos a conocer nuestra motivacion y el marco de trabajo que usamos,capitulo 4 mostramos un experimento para justificar nuestra investigación y en el capitulo 5 mostramos nuestras conclusiones y trabajos futuros
